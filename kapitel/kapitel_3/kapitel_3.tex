\newpage
\section{Methodology}
Regarding requirements analysis, members of the Telekom field-operations team, which consists of
mobile technicians, will be interviewed. The project will just focus on a group of selected potential
stakeholders, as the project is limited in time and resources. This makes it possible to stay in 
control over the time of the project. Literature research will also take a part in the requirements 
analysis. The goal is to use the already available findings productively and to elaborate on
those findings if possible. \\ \\
Prototyping will be another important research method. With this method, the goal is to develop a 
working prototype with just the most important features. This saves the project time and resources,
and also makes it possible to get feedback from the stakeholders as early as possible. \\ \\
Lastly testing will be utilized as a research method. People who are not involved in the development 
project will be asked to test the prototype. The goal is to get feedback on usability and functionality, 
and to get an outside perspective on the project. \\