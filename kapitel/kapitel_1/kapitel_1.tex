\newpage
\section{Einführung in SAP-CO}

Das Controlling ist ein essenzieller Bestandteil eines Unternehmens. Es dient dazu, dem Management wichtige Informationen zur Entscheidungsfindung bereitzustellen.
Diese Entscheidungen wirken sich dabei auf die Koordination, Überwachung und Optimierung aller Unternehmensprozesse aus. \cite{ControllingSAP}

\subsection{Funktionsumfang von SAP-CO}
Das Controlling in SAP ist in verschiedene Module unterteilt. Diese Module sind:

\begin{itemize}
    \item \textbf{Kostenartenrechnung} (CO-OM-CEL) 
    Bei der Kostenartenrechnung werden die verschiedenen Kostenarten eines Unternehmens definiert und zugeordnet um einen Überblick zu erhalten.
    Da Finanzbuchhaltung und Controlling stark miteinander verknüpft sind, kommen viele der Werte direkt aus der Finanzbuchhaltung. \cite{ControllingSAP} \cite{Kostenartenrechnung}

    \item \textbf{Kostenstellenrechnung} (CO-OM-CCA)
    Die Kostenstellenrechnung dient dazu, angefallene Kosten auf die passenden Kostenstellen zu verteilen. So kann überwacht werden an welchen Stellen kosten anfallen 
    und gegebenenfalls entgegengesteuert werden. \cite{ControllingSAP}

    \item \textbf{Prozesskostenrechnung} (CO-OM-ABC)
    Ähnlich wie bei der Kostenstellenrechnung werden bei der Prozesskostenrechnung Kosten verteilt. Hierbei schaut man jedoch anstelle der Kostenstellen auf die verschiedenen Prozesse im Unternehmen.
    Auch hier steht die Optimierung der Abläufe im Vordergrund. \cite{ControllingSAP}

    \item \textbf{Innenaufträge} (CO-OM-OPA)
    Bei diesem Modul können Kosten für interne Projekte oder Aufgaben gesammelt und Kontrolliert werden.\cite{ControllingSAP} \cite{Innenauftrag}

    \item \textbf{Produktkosten-Controlling} (CO-PC)
    Beim Produktkostencontrolling werden die für ein bestimmtes Produkt anfallenden Produktionskosten berechnet und überwacht. Das dient dazu,
     bei der Preisfindung eines Produktes zu unterstützen. \cite{ControllingSAP}

    \item \textbf{Ergebnis- und Marktsegmentrechnung} (CO-PA)
    Bei der Ergebnis- und Marktsegmentrechnung wird betrachtet in welchen Marktsegmenten das Unternehmen wie erfolgreich ist. Das kann dabei helfen
     die Zielgruppe zu identifizieren oder passende Preise zu finden. \cite{ControllingSAP}

    \item \textbf{Profitcenter-Rechnung} (EC-PCA)
    Die Profitcenter-Rechnung dient dazu, die verschiedenen Profitcenter, also bestimmte eigenständige Bereiche, zu bewerten und zu überwachen. \cite{ControllingSAP}

\end{itemize}

