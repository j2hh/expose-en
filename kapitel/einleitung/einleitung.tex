\newpage
\section{Introduction}

\subsection{Background}
“Expert Finding Systems (EFS), also called Expertise Location Systems (ELS) enable users to discover subject matter experts in order to hire or acquire their knowledge.” (\cite[page vii]{maybury_expert_nodate}) It is an important asset especially for big corporations, as it can boost efficiency and lower the barrier of entry for new employees. First appearances of EFS can be traced back to the papers “Enterprise expert and knowledge discovery” (\cite{mattox_enterprise_1999}) and “Facilitating the Online Search of Experts at NASA Using Expert Seeker People-Finder” (\cite{becerra-fernandez_facilitating_nodate}) . Further research has been conducted by Mark T. Maybury in his Paper “Expert Finding Systems” (\cite{maybury_expert_nodate}). The National Forum on Expert Finding Systems states Research Gate, LinkedIn or Harvard Catalyst Profiles as current examples of EFS. (\cite{noauthor_national_nodate})  While examples like Research Gate or LinkedIn are more general approaches to EFS, corporations face the challenge of implementing EFS that fit their specific needs and organizational structures.
Four of the most commonly used organizational structures are the functional structure, which focuses of a clear chain of command and separates the organization into different departments based of their expertise \cite{noauthor_what_nodate-1} , a product- or market-based structure where different departments are based on different products or markets instead of expertise, the geographical structure which divides teams based on their location and a process based structure which groups the employees into teams based on the business processes they are engaging in.   
An alternative approach is the matrix structure. The matrix structure is on the rise with 84\% of employees being “matrixed” in some way according to a study of cross-functional teams conducted by Gallup.   The Matrix organization stands out by having multiple lines of reporting, meaning that employees have two or more bosses effectively.     This makes the Matrix organization a great match for agile working and cross functional teams. The main challenge that needs to be addressed regarding an EFS in a Matrix organization are dynamic and constantly changing tasks and fields of expertise, as people are incentivized to grow in those environments.
Therefore, an EFS has to be able to handle those constant changes in ability, especially because it nearly impossible for the employees to keep track of all their colleagues’ skills over time. 
Technologische Grundlagen (AI, …)
Geschäftlicher Nutzen?
Forschungslücken


\subsection{Problem Statement}
In an Harvard Business Review article, John Ferraro, the former COO of Ernst \& Young suggests, that in order to keep up with the pace of change, companies have to constantly reorganize.  On the other hand, according to a McKinsey survey, over 80\% fail to deliver the hoped-for value in time, with 10\% even causing real damage to the company.  The article also states that two-thirds of company reorganizations do at lease improve the performance to a degree.  This suggests that there is still some room for improvement regarding the performance. 
This room can be utilized by increasing the efficiency of internal processes with an EFS in a few ways. 
This paper evaluates the design and piloting of such an EFS at the Field Operations Mobile (T-FOPS) (Figure 1) team at Deutsche Telekom Technik GmbH (DTT). T-FOPS utilizes a mixture of different organizational structures. On the top-level it is a functionally and partially geographical divided in the sectors Business Operations (T-FOBIZ), Field Operations Mobile CORE (F-FOC), RAN Norddeutschland (T-FORN) and RAN Süddeutschland (T-FORS), the last two meaning North- and South-Germany. The Thesis will focus on T-FOBIZ (Figure 2) which is subdivided functionally into the Digital Unit (T-FOBOD), Operations Support (T-FOBOS), Temporäre Mobilfunkversorgung (T-FOBOT) meaning temporary mobile coverage, and Vergabemanagement (F-FOBOV) meaning procurement management. 

\begin{figure}[H]
    \centering
    \includegraphics[width=0.7\textwidth]{abbildungen/FopsOrga}
    \caption{Organization Chart T-FOPS}
    \label{fig:FopsOrga}
    source: own illustration
\end{figure}
\begin{figure}[H]
    \centering
    \includegraphics[width=0.7\textwidth]{abbildungen/FobizOrga}
    \caption{Organization Chart T-FOBIZ}
    \label{fig:FobizOrga}
    source: own illustration
\end{figure}

\subsection{Structure of the Thesis}

\subsection{Scope and Limitations}

