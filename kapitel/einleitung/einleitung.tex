\section{Einleitung}
Nach Umsatz ist SAP das weltweit drittgrößte Softwareunternehmen. \cite{GroessteUnternehmen} So ist die Software aus der heutigen Wirtschaft nicht mehr wegzudenken.
So setzen Großkonzerne wie die Deutsche Telekom, Bosch oder auch Siemens aber auch kleinere und mittelständische Unternehmen auf die Software von SAP. \cite{SAPCustomers}
Produktkostencontrolling (SAP CO-PC), ein Teilbereich des Controllings (SAP CO), ist ein wichtiger Bestandteil von SAP und wird im Folgenden näher betrachtet.

\subsection{Zielsetzung}
Ziel der Arbeit ist es, einen Einstieg in das Thema Produktkostencontrolling zu geben. Dabei wird kurz auf die Grundlagen des Controllings eingegangen und anschließend das Produktkostencontrolling anhand eines
Fallbeispiels vorgestellt. 

\subsection{Aufbau der Arbeit}
Die Arbeit ist in vier Kapitel unterteilt. Das erste Kapitel ist hierbei die Einleitung welche sich mit der Zielsetzung und dem Aufbau der Arbeit beschäftigt. Im zweiten Kapitel wird auf die Grundlagen des Controllings eingegangen. 
Das dritte Kapitel stellt Produktkostencontrolling anhand einer Fallstudie praktisch vor, und im vierten Kapitel wird das Thema noch einmal kritisch beleuchtet und ein Fazit gezogen.
