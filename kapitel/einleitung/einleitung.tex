\newpage
\section{Introduction}
In an age in which agile working and frequent reorganizations are on the rise 
in large companies, it gets increasingly complicated for teams to stay structured 
and organized. Those challenges can be tackled with a good internal infrastructure 
and company fitted tools.
\\
\\
Despite a high level of digitalization and automation in the business world, there are still 
great challenges to overcome. For many of those challenges, custom software solutions can 
be developed to solve those problems. One of those fields, that can still be optimized is 
the search for suitable contacts within a company. In day-to-day business, it is almost 
daily necessary to work together with colleagues from other departments or teams to tackle 
a variety of problems. Right now, most of the time a handful of colleagues have to be asked
via chat or email to find the right contact. Although tools like Microsoft Teams, Jira or 
Slack offer the possibility to search for colleagues, they are not specialized in finding colleagues
based on a specific topic or problem. This is where this project tries to offer a solution.\\
\\
The following sections will provide a detailed overview of the project goals, technologies as 
well as methodologies that will be used in the course of the project.

\section{Problem Statement}
In the Telekom field operations team, and in many similar teams, it is often necessary to collaborate
with colleagues from other departments and teams. The problem is that often times processes to ensure 
fast and efficient communication are missing. Additionally, frequent reorganizations and agile working 
environments make it difficult to build a network of the right experts. This leads to an unnecessary
loss of time and efficiency. This project aims to solve this problem by offering a custom fitted tool
that supports finding the right contacts by providing a user-friendly tool that specializes in finding 
the right contacts with the use of keywords and topics, as well as an automatic search optimization through
machine learning approaches.

\section{Research interest}
The findings of this paper will show the impact of an optimized knowledge transfer. Time savings, 
efficiency gains and the impact on synergy effects will be presented in the course of the project.
In addition, the prototype that will be developed in course of this project can be used as a template
or basis for companies that want to develop a similar contact management tool and contact database.
When focusing on the scientific aspect of this work, it will show the impact of optimization in this area,
and findings can also be used in related fields, possibly even creating new perspectives in some areas.



