\newpage
\section{Kritischer Ausblick und Fazit}
Nach dem Absolvieren der Fallstudie und der Auseinandersetzung mit dem Thema Produktkostencontrolling wird einem klar, wie viele Möglichkeiten und Funktionen
SAP und insbesondere SAP-CO-PC bietet. Durch die hohe Anpassbarkeit, welche durch ein umfangreiches Angebot an Funktionen und Einstellungen gewährleistet 
wird, kann SAP-CO in nahezu jedem produzierenden Unternehmen sinnvoll eingesetzt werden. Auf der anderen Seite bringt ein solcher Funktionsumfang aber auch eine hohe Komplexität
mit sich, was initial einen hohen Schulungsaufwand bedeutet. Hierfür bietet SAP jedoch auch Schulungen und Zertifikate an. Eine weitere Herausforderung
sind die Daten, welche Sorgfältig gepflegt und eingegeben werden müssen, um ein aussagekräftiges Ergebnis zu erzielen.\\
Mit Blick in die Zukunft gibt es für das Modul CO-PC auf jeden Fall noch Potenzial zur Verbesserung. Mit den aktuellen Entwicklungen im Bereich der Künstlichen Intelligenz und Machine Learning
wäre es denkbar, dass auch SAP in Zukunft mehr auf diese neuen Technologien setzen könnte. Grade bei Dingen wie einer Trendanalyse oder der Optimierung der eigenen Prozesse könnte Künstliche Intelligenz
in Zukunft Abhilfe verschaffen.
\\
\\
Persönlich konnte ich mich durch das Absolvieren anderer Fallstudien zu dem Thema gut in das Modul einarbeiten, was das Erstellen der eigenen Fallstudie deutlich erleichtert hat. So sind 
einige Prozesse mit der Zeit deutlich intuitiver geworden und auch die Navigation innerhalb der Weboberfläche ist mit der Zeit leichter gefallen. Außerdem gab es einige Kleinigkeiten wie 
das Nutzen von Vorlagen, oder das Anpassen der Oberfläche, die mir das Arbeiten erleichtert haben.\\
Das Schreiben dieser Arbeit hat mir dabei 
geholfen, das Thema zu vertiefen und durch das Erstellen der Fallstudie konnte ich das Gelernte auch direkt anwenden.