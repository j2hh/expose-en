
\section*{\langde{Abkürzungsverzeichnis}\langen{List of Abbreviations}}

\begin{acronym}[WYSIWYG]\itemsep0pt %der Parameter in Klammern sollte die längste Abkürzung sein. Damit wird der Abstand zwischen Abkürzung und Übersetzung festgelegt
  \acro{OC}{FOM Online Campus}
  \acro{WYSIWYG}{What you see is what you get}
  \acro{Beispiel}{Nicht verwendet, taucht nicht im Abkürzungsverzeichnis auf}
  \acro{EFS}{Expert Finding System}
  \acro{ELS}{Expertise Location System}
  \acro{DTT}{Deutsche Telekom Technik GmbH}
  \acro{T-FOPS}{Field Operations Mobile}
  \acro{T-FOBIZ}{Business Operations}
  \acro{T-FOC}{Field Operations Mobile CORE}
  \acro{T-FORN}{RAN Norddeutschland}
  \acro{T-FORS}{RAN Süddeutschland}
  \acro{T-FOBOD}{Digital Unit}
  \acro{T-FOBOS}{Operations Support}
  \acro{T-FOBOT}{Temporäre Mobilfunkversorgung}
  \acro{F-FOBOV}{Vergabemanagement}
  \acro{NLP}{Natural Language Processing}
  \acro{AI}{Artificial Intelligence}
  

\end{acronym}